\documentclass{beamer}
%\documentclass[handout]{beamer}

% This file is a solution template for:
% EMSC/PHYS3039 Lectures
\mode<presentation>
{
  \usetheme{Madrid}
  \setbeamercovered{invisible}
\setbeamertemplate{footline}[frame number]{}
\setbeamertemplate{navigation symbols}{}
\setbeamertemplate{footline}{}
}
\usepackage[english]{babel}
\usepackage[latin1]{inputenc}
%\usepackage{times}
\usepackage[T1]{fontenc}

\renewcommand\familydefault{\sfdefault}

\usefonttheme[onlymath]{serif}


\usepackage[cm]{sfmath}

\newcommand{\td}[1]{\frac{D #1}{D t}}
\newcommand{\nd}[2]{\frac{d #1}{d #2}}
\newcommand{\ns}[2]{\frac{d^2 #1}{d #2^2}}
\newcommand{\sd}[2]{\frac{D #1}{D #2}}
\newcommand{\pd}[2]{\frac{\partial #1}{\partial #2}}
\newcommand{\ps}[2]{\frac{\partial^2 #1}{{\partial #2}^2}}
\newcommand{\pt}[2]{\frac{\partial^3 #1}{{\partial #2}^3}}
\newcommand{\pst}[3]{\frac{\partial^2 #1}{\partial #2 \partial #3}}
\newcommand{\ptt}[3]{\frac{\partial^3 #1}{{\partial #2}^2 \partial #3}}
\newcommand{\vc}[1]{\mathbf{#1}}
\newcommand{\mtx}[1]{\vc{\mathsf{#1}}}


\title[EMSC/PHYS3039]{Climate as a Fluid}

\author{$\quad\quad\quad\quad\quad\quad\quad\quad\quad\quad\quad\quad\quad\quad\quad$Kial~Stewart}

\institute{
  $\quad\quad\quad\quad\quad\quad\quad\quad\quad\quad\quad\quad\quad\quad\quad\quad\quad\quad\quad\quad\quad$Research School of Earth Sciences\\
  $\quad\quad\quad\quad\quad\quad\quad\quad\quad\quad\quad\quad\quad\quad\quad\quad\quad\quad\quad\quad\quad$kial.stewart@anu.edu.au\\
  }

\date{$\quad\quad\quad\quad\quad\quad\quad\quad\quad\quad\quad\quad\quad\quad\quad$Semester 1, 2024}

%%%%%%%%%%%%%%%%%%%%%%%%%%%%%%%

\begin{document}

\begin{frame}
  \titlepage
\end{frame} 

%%%%%%%%%%%%%%%%%%%%%%%%%%%%%%%


  \frame{
  \frametitle{$\quad$Climate as a Fluid}

}
  
  
  %%%%%%%%%%%%%%%%%%%%%%
  
  \frame{
  \frametitle{$\quad$Climate System}

\begin{columns}
\column{0.5\textwidth}
%\center{\includegraphics[width=1.0\hsize]{One-Does-Not-Simply-m1zju.jpg}}
\column{0.5\textwidth}
%\center{\includegraphics[width=0.7\hsize]{B5V0m_yCMAEnZHy.jpg}}\\
\end{columns}
}


  %%%%%%%%%%%%%%%%%%%%%%
  
  \frame{
  \frametitle{$\quad$Job of the Climate}


}


  %%%%%%%%%%%%%%%%%%%%%%
  
  \frame{
  \frametitle{$\quad$Earth's Radiation Budget}


}



  %%%%%%%%%%%%%%%%%%%%%%
  
  \frame{
  \frametitle{$\quad$Balancing Earth's Radiation Budget}


}




%  
%  %%%%%%%%%%%%%%%%%%%%%%
%  
%  \frame{
%  \frametitle{Question: Rossby waves in lee of topography}
%\center{\includegraphics[width=0.65\hsize]{rossbyleewavesketch_cropped.pdf}}
%
%Imagine eastward flow encountering a topographic ridge.
%
%Sketch the evolution of the flow as it crosses the ridge and downstream of the ridge.
%
%
%%Rossby waves in eastward flow (Northern hemisphere), generated by
%%compression/stretching of vortex columns over the topography.   
%%North-South gradient of $f$ supports Rossby waves downstream.
%
%}
%






\end{document}







































